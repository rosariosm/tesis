% Comandos personalizados

% {\fechaPresentacion} :: para escribir la fecha de presentación del trabajo
\newcommand{\fechaPresentacion}{\today}
% {\unlp} :: para escribir "Universidad Nacional de La Plata"
\newcommand{\unlp}{Universidad Nacional de La Plata}
% {\facultad} :: para escribir "Facultad de Informática"
\newcommand{\facultad}{Facultad de Informática}
% {\cespi} :: para escribir "CeSPI"
\newcommand{\cespi}{CeSPI}
% {\direccionDesarrollo} :: Para escribir "Dirección de Desarrollo del CeSPI"
\newcommand{\direccionDesarrollo}{Dirección de Desarrollo del {\cespi}}
% {\tituloTrabajo} :: Para escribir el título de la tesina
\newcommand{\tituloTrabajo}{NutriBuddy, tu mejor amigue}
% \tituloTrabajoDosLineas :: Para escribir el título de la tesina en dos líneas (carátula)
\newcommand{\tituloTrabajoDosLineas}{NutriBuddy \\* tu mejor amigue}
% {\cloud} :: para escribir el nombre clave de la nueva nube
\newcommand{\cloud}{Cloud}
% {\oaispec} :: para escribir "OpenAPI-Spec"
\newcommand{\oaispec}{OpenAPI-Spec}
% {\rsantamarina} :: para escribir "Rosario Santa Marina"
\newcommand{\rsantamarina}{Rosario Santa Marina}
% {\santamarinar} :: para escribir "Santa Marina, Rosario"
\newcommand{\santamarinar}{Santa Marina, Rosario}
% {\jrey} :: para escribir "Julieta Rey Crespo"
\newcommand{\jrey}{Julieta Rey Crespo}
% {\reyj} :: para escribir "Rey Crespo, Julieta"
\newcommand{\reyj}{Rey Crespo, Julieta}

% \eng{English expression} :: para denotar que "English expression" está en inglés
\newcommand{\eng}[1]{\textit{#1}}

% {\caratula} :: para generar la carátula de la tesina
\newcommand{\caratula}{
  \begin{center}
    \includegraphics{src/images/caratula/unlp.png}\\
    \huge{\unlp}\\
    \vspace{5mm}
    \huge{\facultad}\\
    \vspace{5mm}
    \large{Tesina de la Licenciatura}\\
    \vspace{15mm}
    \huge{\tituloTrabajoDosLineas}\\
    \vspace{10mm}
    \large{\textbf{\reyj} \\
    \textbf{\santamarinar}}\\    
    \vspace{20mm}
    \large{Directoras: Harari, Ivana y Fava, Laura}\\
    \vspace{20mm}
    \normalsize{\fechaPresentacion}\\
  \end{center}
}

% {\checkmark} :: para imprimir un check (tilde)
\def\checkmark{\tikz\fill[scale=0.4](0,.35) -- (.25,0) -- (1,.7) -- (.25,.15) -- cycle;}

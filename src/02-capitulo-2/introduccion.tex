Este capítulo tiene como objetivo mostrar el marco teórico del cual se fundamenta esta tesina para brindarle un contexto al lector del presente trabajo. 

En primer lugar, se definirá qué es el juego y para qué sirve. Luego se explicará que los juegos no son sólo para divertirse y que existen juegos serios que se utilizan para la enseñanza, haciendo hincapié en los videojuegos. Se analizará el sistema escolar actual y cómo se puede educar jugando, en particular en la escuela secuntaria. Por último se verá como CodeCombat resulta una herramienta útil para este contexto.


NUTRICION 

	a. Beneficios de una alimentacion balanceada
	b. Distintos tipos de objetivos
	c. Profundizar en la nutricion y aspectos psicológicos
    d. Trastornos de alimentacion orientado a nuestra tesis
    e. Cambio de hábitos y dificultades para sostenerlo en el tiempo
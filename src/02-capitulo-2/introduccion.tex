Este capítulo tiene como objetivo mostrar el marco teórico del cual se fundamenta esta tesina para brindarle un contexto al lector del presente trabajo. 

En primer lugar, se definirá qué es el juego y para qué sirve. Luego se explicará que los juegos no son sólo para divertirse y que existen juegos serios que se utilizan para la enseñanza, haciendo hincapié en los videojuegos. Se analizará el sistema escolar actual y cómo se puede educar jugando, en particular en la escuela secuntaria. Por último se verá como CodeCombat resulta una herramienta útil para este contexto.


NUTRICION 

	a. Beneficios de una alimentacion balanceada
	b. Distintos tipos de objetivos
	c. Profundizar en la nutricion y aspectos psicológicos
    d. Trastornos de alimentacion orientado a nuestra tesis
    e. Cambio de hábitos y dificultades para sostenerlo en el tiempo


1. Qué es la nutrición 

Aca irian pequeñas definiciones de distintos autores y aalgo muy genérico

https://www.who.int/es/news-room/fact-sheets/detail/healthy-diet

1.1. Importancia en el desarrollo de los individuos y cómo impacta en la sociedad
Acá me gustaría meter datos duros, cómo impacta en la sociedad si los chicos se alimentan mal vs los que tienen privilegios. Qué pasa despues con esa sociedad, quienes acceden  a la nutrición y qué beneficios les trae. Por qué este tema tiene que ser super importante en una sociedad

1.2. Apectos psicológicos de la nutrición
  - Cambios de hábitos y cómo se sostienen en el tiempo
  - Trastornos alimenticios

1.3. Dieta equilibrada
	- Importancia de los alimentos 
	- Dietas en función de objetivos
	- Actividad física


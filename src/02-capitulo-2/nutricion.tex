\subsection{¿Qué es la nutrición?}
\label{nutrición}


La nutrición nos acompaña desde que nos desarrollamos, durante toda nuestra infancia, adolescencia, adultez y vejez. Es esencial en nuestra vida y un pilar fundamental de la buena salud. Es una disciplina científica que se focaliza en el acceso y la utilización de la comida y nutrientes con el objetivo de mejorar la vida diaria, la salud, el crecimiento, el desarrollo y el bienestar de las personas\cite{nutritionForHealthAndDevelopment}. La ingesta de alimentos está íntimamente relacionada con las necesidades dietéticas del organismo de los individuos, no es lo mismo una persona que hace mucha actividad física a una persona más sedentaria. Una buena nutrición consiste en una dieta suficiente y equilibrada combinada con ejercicio regular.

Desde la perspectiva de los derechos humanos podemos definir a la nutrición como:

\begin{itemize}
  \item La Declaración Universal de los Derechos Humanos (1948) dice que "Toda persona tiene derecho a un nivel de vida adecuado que le asegure, asi como a su familia, la salud y el bienestar, y en especial la alimentación..." \cite{declaracionUniversalDDHH}
  \item Another entry in the list
\end{itemize}


. Toda persona tiene derecho a un nivel de vida adecuado que le asegure, así como a su familia, la salud y el bienestar, y en especial la alimentación,














Desde el sector salud se percibe a la desnutrición como multicausal y compleja, en la que interactúan factores socioeconómicos, culturales y psicológicos, asociados a la causa principal que es la falta de recursos económicos suficientes para realizar una alimentación equilibrada. Pobreza-infección-desnutrición aparecen íntimamente relacionadas.

Una mala nutrición puede reducir la inmunidad, aumentar la vulnerabilidad a las enfermedades, alterar el desarrollo físico y mental, y reducir la productividad.

\subsection{¿Qué es la nutrición?}
\label{nutrición}


La nutrición nos acompaña desde que nos desarrollamos, durante toda nuestra infancia, adolescencia, adultez y vejez. Es esencial en nuestra vida y un pilar fundamental de la buena salud. Es una disciplina científica que se focaliza en el acceso y la utilización de la comida y nutrientes con el objetivo de mejorar la vida diaria, la salud, el crecimiento, el desarrollo y el bienestar de las personas\cite{nutritionForHealthAndDevelopment}. La ingesta de alimentos está íntimamente relacionada con las necesidades dietéticas del organismo de los individuos, no es lo mismo una persona que hace mucha actividad física a una persona más sedentaria. Una buena nutrición consiste en una dieta suficiente y equilibrada combinada con ejercicio regular.

Es tan importante que hasta aparece en diversos documentos que tratan los derechos humanos:

\begin{itemize}
  \item La Declaración Universal de los Derechos Humanos (1948) define en el artículo 25 que "Toda persona tiene derecho a un nivel de vida adecuado que le asegure, asi como a su familia, la salud y el bienestar, y en especial la alimentación..." \cite{declaracionUniversalDDHH}
  \item En la Convención de los Derechos del Niño en dos artículos se nombra a la nutrición. En el artículo 24 determina que los niños deben un alto estándar de salud y los estados deben tomar las medidas necesarias para combatir la enfermedad y la malnutrición a través de facilitar nutrientes adecuados y agua segura\cite{convencionDerechosDelNiño}.
  \item La Declaración Mundial de Nutrición afirma que el acceso a una nutrición adecuada y comida saludable es un derecho de todos los individuos.\cite{declaracionMundialDeNutricion}.
\end{itemize}

No se puede hablar de nutrición sin mencionar la desnutrición. Esta se percibe como multicausal y compleja, en la que interactúan diferentes factores socioeconómicos, culturales y psicológicos. Debido a que la causa principal de la desnutrición es la falta de recursos económicos suficientes para realizar una alimentación equilibrada, la pobreza-infección-desnutrición aparecen íntimamente relacionadas. Una mala nutrición puede reducir la inmunidad, aumentar la vulnerabilidad a las enfermedades, alterar el desarrollo físico y mental, y reducir la productividad\cite{malnutricionInfantil}.

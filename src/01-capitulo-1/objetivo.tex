\subsection{Objetivo}
\label{objetivo}

El objetivo central de esta tesina es aplicar metodologías de desarrollo centrado en el usuario para realizar una herramienta que permita dar seguimiento a planes de nutrición y cambios de hábitos, haciendo foco en el refuerzo positivo para mantener el engagement de los usuarios.

Se espera mejorar y agilizar la comunicación nutricionista-paciente al ofrecer la posibilidad de automatizar el envío de estudios médicos, planes nutricionales, carga de comidas, entre otros datos. También se busca promover los buenos hábitos alimenticios educando y mostrando el progreso del usuario obtenido a lo largo del tiempo. 

Además, se harán pruebas con distintos perfiles de usuarios para pulir la experiencia de uso a través de distintas iteraciones y de esta manera validar su potencial. El propósito es simplificar la interfaz al máximo para que sea intuitiva, cómoda, accesible e inclusiva, de manera tal que el usuario pueda generar un vínculo e incorpore la herramienta en su vida diaria.

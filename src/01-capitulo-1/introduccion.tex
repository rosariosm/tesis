Es sabido que una vida sedentaria acompañada de mala alimentación deterioran la calidad de vida. El cambio de paradigma al que la sociedad se vio expuesta en el año 2020 a raíz del COVID-19 tampoco colaboró con mejorar esta situación. En este contexto muchas personas dejaron de hacer sus consultas médicas, los gimnasios dejaron de ser un lugar seguro y las personas dejaron de movilizarse como lo hacían antes. Cambió la forma de vivir de la sociedad y en cosecuencia sus hábitos nutricionales y físicos.

Existen muchas aplicaciones en el mercado que tratan de suplir estas falencias pero traen nuevas dificultades: la búsqueda individual, autónoma y dispersa sumado a la falta de supervisión de un profesional puede terminar siendo nociva. Son aplicaciones impersonales, no se conoce quiénes están dando consejos e ignoran las características particulares de cada paciente. Además, estas aplicaciones suelen enfocarse sólo en los individuos que quieren cambiar su dieta y no le facilitan la tarea a los profesionales de la nutrición, quienes hoy en día siguen manteniendo métodos no-digitales para realizar sus tareas diarias.

La generación de hábitos saludables se alcanza mediante la constancia, algo que dado este contexto y las particularidades psicológicas de las personas, puede ser muy dificil de alcanzar de forma individual. Tambien es necesario contar con el asesoramiento de un profesional para evitar exponernos a cualquier tipo de desbalance nutricional. Por esta razón resulta indispensable pensar una herramienta al alcance del bolsillo que permita una conexión ágil con un profesional de confianza, que permita dar seguimiento a los objetivos y estar alerta de cualquier conducta nociva. Evitar la deserción es un factor clave. Incentivar el progreso, reconocer logros y brindar un espacio de seguimiento y contención son fundamentales para que el paciente logre los objetivos propuestos. 

Teniendo en cuenta el cambio de paradigma que estamos transitando, la falta de tecnologías serias aplicadas al ámbito de nutrición,la necesidad de tecnología aplicadas que fomente el vínculo paciente-nutricionista y la deserción en los tratamientos para cambios de hábitos alimenticios, nace NutriBuddy. En este tesina se desarrollará, aplicando metodologías de diseño centrado en el usuario, una herramienta que permitirá, mediante una interfaz innovadora, usable y accesible, acompañar a los usuarios en su recorrido por los cambios de hábitos alimenticios y facilitar el seguimiento de los pacientes por parte de un nutricionista.


\cite{educacionPrimaria2018}
El aprendizaje mediante el juego es clave para el correcto desarrollo cognitivo. En nuestros primeros años de vida se le da vital importancia pero a medida que vamos creciendo esta actividad se va dejando de lado poco a poco y se relega solo a espacios de ocio, entretenimiendo y diversión. En la educación artentina, durante el Nivel Inicial (desde los 45 días hasta los 5 años de vida), la enseñanza se centra en actividades lúdicas que permiten a los infantes conocerse a sí mismos y comprender el entorno en el que viven, construyendo su identidad como seres sociales. ¿Pasa lo mismo a partir del Nivel Primario?

El uso de juegos educativos es un enfoque ideal para complementar la educación tradicional ya que son atractivos e innovadores y rompen con la rutina diaria; enseñan en un ambiente divertido y relajado; reducen la brecha entre estudiantes que tienen facilidades para aprender nuevos conocimientos de los que no; y permiten resolver problemas de forma colaborativa mejorando las relaciones interpersonales.

La educación en el mundo está cambiando y adaptándose a las nuevas tecnologías, y Argentina no se queda atrás: gracias a las TICs (Tecnologías de la Información y la Comunicación) en los nuevos diseños curriculares aparecen recursos extra para complementar la educación tradicional. Por ejemplo, en el área de Matemática \cite{educacionPrimaria2018} se sugiere utilizar videojuegos lógico-matemáticos para resolver problemas ya que un entorno lúdico, al explotar la experimentación como parte del proceso resolutivo, facilita la exploración de diferentes soluciones.

Teniendo en cuenta la tendencia de enseñar programación en las escuelas, incluso desde el Nivel Primario, la utilización de videojuegos educativos no es trivial. En esta tesina se desarrollará una adaptación de la herramienta open source CodeCombat para enseñar Python en las escuelas secundarias argentinas. Esta herramienta permite enseñar conceptos de la programación en un ambiente atractivo y amigable, ofreciendo las ventajas de los videojuegos y la enseñanza lúdica, y cumpliendo con lo establecido en la educación argentina. 